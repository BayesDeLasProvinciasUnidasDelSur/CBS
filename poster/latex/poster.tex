\documentclass[a0,portrait]{a0poster} %A0 841mm x 1189mm
\usepackage{multicol} % This is so we can have multiple columns 
\columnsep=80pt % This is the amount of white space between the columns in the poster
\columnseprule=3pt % This is the thickness of the black line between the columns in the poster
\usepackage[svgnames]{xcolor} % Specify colors by their 'svgnames', for a full list of all colors available see here: http://www.latextemplates.com/svgnames-colors
\usepackage[makeroom]{cancel} % \cancel{} \bcancel{} etc
\usepackage{multirow}
%\usepackage{times} % Use the times font
%\usepackage{palatino} % Uncomment to use the Palatino font
%\usepackage[sfdefault]{AlegreyaSans}
\usepackage[sfdefault]{AlegreyaSans}
\usepackage{graphicx} % Required for including images
\graphicspath{{figures/}} % Location of the graphics files
\usepackage{booktabs} % Top and bottom rules for table
\usepackage[labelfont=bf]{caption} % Required for specifying captions to tables and figures
\captionsetup[subfigure]{font=Large}
\captionsetup[figure]{font=Large}
\usepackage{amsfonts, amsmath, amsthm, amssymb} % For math fonts, symbols and environments
\usepackage{wrapfig} % Allows wrapping text around tables and figures
\usepackage{bm}
\usepackage{ragged2e}
\usepackage{float} % para que los gr\'aficos se queden en su lugar con [H]
\usepackage[subrefformat=parens]{subcaption} % para \begin{subfigure}
\usepackage{tikz} % Para graficar, por ejemplo bayes networks
\usepackage{framed}
\usepackage{mdframed}
\usepackage[utf8]{inputenc}
  
\usepackage{lipsum} % Para rellenar con texto dummy.
\usepackage[absolute,overlay]{textpos} % Para
\setlength{\TPHorizModule}{1cm} %
\setlength{\TPVertModule}{1cm}	%

\usetikzlibrary{bayesnet} % Para que ande se necesita copiar el archivo  tikzlibrarybayesnet.code.tex en la misma carpeta

% Para escribir en un mismo archivo en inglés y en español
\newif\ifen
\newif\ifes
\newcommand{\en}[1]{\ifen#1\fi}
\newcommand{\es}[1]{\ifes#1\fi}
\newcommand{\En}[1]{\ifen#1\fi}
\newcommand{\Es}[1]{\ifes#1\fi}
\estrue % El idioma que compila.

\newcommand{\vm}[1]{\mathbf{#1}}
\newcommand{\N}{\mathcal{N}}
\newcommand\hfrac[2]{\genfrac{}{}{0pt}{}{#1}{#2}} %\frac{}{} sin la linea del medio

% \usepackage{xr}
% \externaldocument{supplementary}
\setlength{\columnseprule}{0pt}


\addtolength{\textwidth}{40pt}
\addtolength{\oddsidemargin}{-40pt}

\addtolength{\topmargin}{-1.5cm}
% \addtolength{\textwidth}{10pt}
\addtolength{\textheight}{4cm}
% \addtolength{\oddsidemargin}{-5pt}

\begin{document}

\centering \fontsize{90}{90} \textbf{El título que resume la idea} \\[0.5cm]  % Title
\LARGE \textbf{Nombre Apellido}$^{1,2}$  \ \ \  \texttt{correo@edu.ar} \\
\large 1. Afiliación institucional. Santiago del Estero, Argentina \\
\large 2. Otra afiliación institucional.\\


\vspace{1cm}

\includegraphics[width=0.2\linewidth]{../../logos/CBP.png} \\
\fontsize{55}{55}\selectfont El logo de tu institución \\[0.3cm]



%\begin{paracol}{2}
\begin{multicols}{2}
% This is how many columns your poster will be broken into, a portrait poster is generally split into 2 columns

% Tamaño del texto
\fontsize{40}{50}\selectfont






















\centering

\justify 

\textbf{\En{Summary}\Es{Resumen}.} \lipsum[2]


\vspace{1cm}
\textbf{Introduc\en{tion}\es{ción}.} \lipsum[2]


\vspace{1cm}
\textbf{Métodos.} El modelo probabilístico.

\vspace{0.3cm}

\begin{figure}[H]  \centering
\begin{subfigure}[b]{0.39\linewidth}
  \centering
  \scalebox{1.5}{
   \tikz{
    \node[det, fill=black!10] (r) {$r$} ;
    \node[const, below=of r, yshift=-1.5cm] (ir) {} ;
    \node[const, right=of r] (dr) {$ r = (d > 0)$};
    \node[const, below=of dr, xshift=-.1cm] (r_name) {\small \en{Result}\es{Resultado}};


    \node[latent, above=of r, yshift=-1cm] (d) {$d$} ; %
    \node[const, right=of d] (dd) {$ d = p_i-p_j$};
    \node[const, below=of dd, xshift=-.1cm] (d_name) {\small \en{Difference}\es{Diferencia}};

    \node[latent, above=of d, xshift=-1.5cm, yshift=-1cm] (p1) {$p_i$} ; %
    \node[latent, above=of d, xshift=1.5cm, yshift=-1cm] (p2) {$p_j$} ; %


    \node[accion, above=of p1,yshift=0.5cm] (s1) {} ; %
    \node[const, right=of s1] (ds1) {$s_i$};
    \node[accion, above=of p2,yshift=0.5cm] (s2) {} ; %
    \node[const, right=of s2] (ds2) {$s_j$};

    \node[const, right=of p2] (dp2) { $p \sim \N(s,\beta^2)$};
    \node[const, below=of dp2, xshift=-0.1cm] (p_name) {\small \en{Performance}\es{Desempeño}};

    \node[const, right=of s2, xshift=2cm] (s_name) {\small \en{Skill}\es{Habilidad}};

    \edge {d} {r};
    \edge {p1,p2} {d};
    \edge {s1} {p1};
    \edge {s2} {p2};
    }
  }
  \end{subfigure}
  \caption{Hola}
\end{figure}


\lipsum[4]


Para comparar modelos necesitamos computar el bayes factor (BF),
\begin{equation}\label{eq:bayes_factor}
\begin{split}
\frac{P(\text{Model\es{o}}_i|\text{Dat\en{a}\es{os}})}{P(\text{Model\es{o}}_j|\text{Dat\en{a}\es{os}})} = \frac{P(\text{Dat\en{a}\es{os}}|\text{Model\es{o}}_i)\cancel{P(\text{Model\es{o}}_i)}}{P(\text{Dat\en{a}\es{os}}|\text{Model\es{o}}_j)\cancel{P(\text{Model\es{o}}_j)}}
\end{split}
\end{equation}\\[-0.4cm]
Entonces, la media geométrica (GM) es la tasa de crecimiento de largo plazo que induce la probabilidad de los modelos alternativos,
\vspace{0.4cm}
\begin{equation*}
\begin{split}
P(\text{Dat\en{a}\es{os}}|\text{Model\es{o}}) & = P(d_1|\text{Model\es{o}})P(d_2|d_1,\text{Model\es{o}}) \dots \\
& = \text{geometric mean}(P(\text{Dat\en{a}\es{os}}|\text{Model\es{o}}))^{|\text{Dat\en{a}\es{os}}|}
\end{split}
\end{equation*}

\vspace{0.5cm}

\textbf{Resultados.} El modelo X logra resultados similares y hasta mejores que modelos más complejos como Y, de forma más eficiente:
\vspace{0.3cm}
\begin{table}[H] \centering
\normalsize
\scalebox{1.15}{
  \begin{tabular}{|c|c|c|c|c|c|c|}
  \hline
  1 & 2 & 3 & 4 & 5 & 6 & 7 \\ \hline
  Uno & Dos & Tres & Cuatro & Cinco & Seis & Siete \\ \hline
  \end{tabular}
  }
\end{table}




%%%%%%%%%%%%%%%%%%%%%%%%%%%%%%%%%%%%%%%%%%%%%%%%%%%%%%%%%%%%%%%%%%%%%%%%%%%%%%%%%%
\columnbreak
%%%%%%%%%%%%%%%%%%%%%%%%%%%%%%%%%%%%%%%%%%%%%%%%%%%%%%%%%%%%%%%%%%%%%%%%%%%%%%%%%%%



\justify

\lipsum[4]

\vspace{1cm}
\begin{figure}[H]
\centering
\begin{subfigure}[b]{1\linewidth}
    \centering
    \includegraphics[width=.4\linewidth]{../../logos/sae}
\end{subfigure}
\caption{Un gráfico}
\end{figure}

\lipsum[5]


\vspace{1cm}
\textbf{Conclution.}
\lipsum[5]


\vspace{1cm}
{ \small
\nocite{*} % Print all references regardless of whether they were cited in the poster or not
\bibliographystyle{plos2015} % Plain referencing style
\bibliography{biblio.bib} % Use the example bibliography file sample.bib
}


\end{multicols}



\end{document}
